\section{Akustische Antenne}
	Eine akustische Antenne ist prinzipiell ein Array aus mehreren Mikrofonen, die in beliebiger Art und Weise angeordnet sein können.
	Die Mikrofone sind dabei so verschaltet, dass sich ein Ausgangssignal für das gesamte Array ergibt.
	Mikrofonarrays können eingesetzt werden, um bestimmte Richtcharakteristika zu erzielen, also beispielsweise den Schall aus einer bestimmten Einfallsrichtung stärker übertragen als Schall der aus anderen Richtungen einfällt.
	In der Messtechnik wird diese Eigenschaft genutzt, um Quellen von Störschall ausfindig zu machen. %(Beispielbild aus Möser??)
	Aber auch in Consumer-Endgeräten wird Arraytechnologie genutzt, zum Beispiel für Mikrofone in Handys oder Laptops.
	Oft wird auch der Begriff "akustische Kamera" synonym verwendet\footnote{Möser (Hrsg.), Messtechnik der Akustik, Springer 2010, S. 365}.