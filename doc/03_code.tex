\section{Implementierung}
	
	Der Delay \& Sum Algorithmus wurde in der Programmiersprache Python\footnote{Version 3.5.3} implementiert. Benötigt wird neben einer entsprechenden Python-Installation auch noch das Package "numpy"\footnote{Version 1.12.1}. Unter debianbasierten Linuxdistributionen (wie zum Beispiel Ubuntu) kann Python über die Paketquellen und numpy mit "Pip" installiert werden, für Windows empfiehlt sich eine Python-Distribution wie zum Beispiel "Anaconda"\footnote{https://www.continuum.io/downloads}.
	
	\subsection{Package delay\_and\_sum}
	
	Das Package enthält die Dateien
	\begin{itemize}
		\item \texttt{delay\_and\_sum.py}
		\item \texttt{signal\_processing.py}
		\item \texttt{\_helper.py}
	\end{itemize}
	In \texttt{delay\_and\_sum.py} wird die Klasse \texttt{DelayAndSumPlane} definiert, die den Delay \& Sum Algorithmus für ebene Wellen umsetzt. Die Methode \texttt{make\_rms\_list} gibt dabei eine Liste mit Effektivwerten für verschiedene Winkel zurück, wie in Gleichung \ref{eq:das_plane} beschrieben.
	Die Klasse \texttt{SignalProcessor} in \texttt{signal\_processing.py} stellt dafür verschiedene Methoden für die Signalverarbeitung zur Verfügung.
	
	Zusäzlich gibt es das \texttt{\_helper.py} Modul, das neben einer Klasse zur Testsignalerzeugung auch Konstanten wie die Schallgeschwindigkeit und Faktoren zur Umrechnung zwischen Bogenmaß und Winkel enthält.